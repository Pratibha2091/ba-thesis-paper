% !TEX root = ../main.tex

\chapter{Related Work}
\label{chap:Related Work}

\section{Newsvendor model}
\label{sec:Newsvendor model}

The newsvendor model~\autocite{Arrow1974} is a mathematical model resembling newsvendor stand. The problem is characterized by fixed prices and uncertain demand for a perishable product, e.g. yesterdays newspapers hold no value today. Demand for newspapers is uncertain and newsvendor must decide how many newspaper to buy for reselling.


The stock in newsvendor model is only for one day, and unlike problem presented in chapter~\ref{chap:prob-def} product is not perishable, but its cost increases in each following time step.

\section{Dynamic lot sizing model}
\label{sec:Dynamic lot sizing model}

Dynamic lot-size model~\autocite{Wagner2004} is generalization of the Economic Order quantity model~\autocite{Harris1990} which takes into account that demand for the product varies over time. For the planning horizon of $n$ time periods we have following data:

\begin{align*}
  d_t && \text{Demand at time period $t$} \\
  h_t && \text{Holding cost at time period $t$} \\
  K_t && \text{Setup cost at time period $t$} \\
  I_0 && \text{Initial inventory} \\
\end{align*}

and decision variable $\mathbf{x}$:
\begin{align*}
  x_t && \text{Quantity purchased at time period $t$}\\
\end{align*}

For simplicity we define inventory at time period $t$ as:

\begin{align*}
  I_t &= I_0 + \sum_{t=0}^k{x_t} - \sum_{t=0}^k{d_t}\\
\end{align*}

And we want to choose optimal $x_t$, under following constraints:

\begin{align*}
  x_t &\ge 0 \; \forall t\\
  I_t &\ge 0 \; \forall t\\
\end{align*}

And we want to minimize following objective function:
\begin{equation*}
  f = \sum_t{h_t I_t + H(x_t)K_t}
\end{equation*}

where $H$ is Heaviside step function.

\subsection{Dynamic lot size model in stochastic setting}
\label{sub:Dynamic lot size model in stochastic setting}

Various approaches exist to handle dynamic lot sizing model in stochastic setting. The most comprehensive analysis can be found in textbook~\autocite{tempelmeier2013stochastic}. Other approaches use rolling horizon for predicting demand quantity~\autocite{cao2013adaptive} and apply modified dynamic lot sizing model by introducing backlogging, that is allowing inventory, $I_t$ to be negative with known penalty, $b_t$ per unit per time step.

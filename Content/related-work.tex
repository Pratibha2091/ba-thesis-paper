% !TEX root = ../main.tex

\chapter{Related Work}
\label{chap:Related Work}

\section{Newsvendor model}
\label{sec:Newsvendor model}

The newsvendor model~\autocite{Arrow1974} is a mathematical model resembling newsvendor stand. The problem is characterized by fixed prices and uncertain demand for a perishable product, e.g. yesterdays newspapers hold no value today. Demand for newspapers is uncertain and newsvendor must decide how many newspaper to buy for reselling.


The stock in newsvendor model is only for one day, and unlike problem presented in chapter~\ref{chap:prob-def} product is not perishable, but its cost increases in each following time step.

\section{Economic order quantity}
\label{sec:EOQ}

Economic order quantity (EOQ) is the order quantity that minimizes the total holding costs and ordering costs. It is one of the oldest classical production scheduling models.~\autocite{Harris1990}

EOQ applies only when demand for a product is constant over the year and each new order is delivered in full when inventory reaches zero. There is a fixed cost for each order placed, regardless of the number of units ordered. There is also a cost for each unit held in storage, commonly known as holding cost.

We want to determine the optimal number of units to order so that we minimize the total cost associated with the purchase, delivery and storage of the product.


\section{Dynamic lot-size model}
\label{sec:Dynamic lot-size model}

Dynamic lot-size model~\autocite{Wagner2004} is generalization of the EPQ mode
which takes into account that demand for the product varies over time. For the planning horizon of $n$ time periods we have following data:

\begin{align*}
  d_t && \text{Demand at time period $t$} \\
  h_t && \text{Holding cost at time period $t$} \\
  K_t && \text{Setup cost at time period $t$} \\
  I_0 && \text{Initial inventory} \\
\end{align*}

and decision variable $\mathbf{x}$:
\begin{align*}
  x_t && \text{Quantity purchased at time period $t$}\\
\end{align*}

For simplicity we define inventory at time period $t$ as:

\begin{align*}
  I_t &= I_0 + \sum_{t=0}^k{x_t} - \sum_{t=0}^k{d_t}\\
\end{align*}

And we want to choose optimal $x_t$, under following constraints:

\begin{align*}
  x_t &\ge 0 \; \forall t\\
  I_t &\ge 0 \; \forall t\\
\end{align*}

And we want to minimize following objective function:
\begin{equation*}
  f = \sum_t{h_t I_t + H(x_t)K_t}
\end{equation*}

where $H$ is Heaviside step function.

% !TEX root = ../main.tex

\chapter{Problem definition}
\label{chap:prob-def}
\section{introduction}

This problem is a variation of the lot sizing problem in a stochastic setting. Variation is as follows. Firstly we describe deterministic variant before extending the model to stochastic setting. We have predictions for raw material supply cost $\mathbf{s}$ at future time moment $t$. The \emph{factory} converts all bought raw materials at time $t$ and stores them in a warehouse. Number of discrete time moments under consideration is denoted by $n$.

One time moment storage costs fixed amount $h$. The  holding cost is per unit and per time period.

Each time moment, $t$, we have certain demand we need to satify, denoted as $d_t$. In case there's no products in storage to satisfy demand we allow backlogging incurring a cost denoted as $b$ per unit per day, for late orders.

Our aim is to optimize our procurement policy by choosing $x_t$, which is amount of the product we buy at time moment $t$. However we are constrained by the maximum amount raw materials , $\mathbf{x^{\text{(max)}}}$, that we can buy each day.

\section{Related Work}

\subsection{Newsvendor model}
\label{sec:Newsvendor model}

The newsvendor model~\autocite{Arrow1974} is a mathematical model resembling newsvendor stand. The problem is characterized by fixed prices and uncertain demand for a perishable product, e.g. yesterdays newspapers hold no value today. Demand for newspapers is uncertain and newsvendor must decide how many newspaper to buy for reselling.


The stock in newsvendor model is only for one day, and unlike problem presented in chapter~\ref{chap:prob-def} product is not perishable, but its cost increases in each following time step.

\subsection{Dynamic lot sizing model}
\label{sec:Dynamic lot sizing model}

Dynamic lot-size model~\autocite{Wagner2004} is generalization of the Economic Order quantity model~\autocite{Harris1990} which takes into account that demand for the product varies over time. For the planning horizon of $n$ time periods we have following data:

\begin{align*}
  d_t && \text{Demand at time period $t$} \\
  h_t && \text{Holding cost at time period $t$} \\
  K_t && \text{Setup cost at time period $t$} \\
  x^{(h)}_0 && \text{Initial inventory} \\
\end{align*}

and decision variable $\mathbf{x}$:
\begin{align*}
  x_t && \text{Quantity purchased at time period $t$}\\
\end{align*}

plus auxiliary variable $y_t$:
\begin{equation*}
    y_t = \begin{cases}
        1 & x_t > 0\\
        0 & x_t = 0 \\
    \end{cases}
\end{equation*}

For simplicity we define inventory at time period $t$ as:

\begin{align*}
  I_t &= x^{(h)}_0 + \sum_{t=0}^k{x_t} - \sum_{t=0}^k{d_t}\\
\end{align*}

And we want to choose optimal $x_t$, under following constraints:

\begin{align*}
  x_t &\ge 0 \; \forall t\\
  I_t &\ge 0 \; \forall t\\
\end{align*}

And we want to minimize following objective function:
\begin{equation*}
  f = \sum_t{h_t I_t + y_t K_t}
\end{equation*}


\subsubsection{Dynamic lot size model in stochastic setting}
\label{sub:Dynamic lot size model in stochastic setting}

Various approaches exist to handle dynamic lot sizing model in stochastic setting. The most comprehensive analysis can be found in textbook~\autocite{tempelmeier2013stochastic}. Other approaches use rolling horizon for predicting demand quantity~\autocite{cao2013adaptive} and apply modified dynamic lot sizing model by introducing backlogging, that is allowing inventory, $I_t$ to be negative with known penalty, $b_t$ per unit per time step.
\section{Formal problem definition}
\label{sec:prob-def}

\subsection{Definitions}
\label{sub:Definitions}

Following is the deterministic problem variant, and in subsequent chapters randomness and uncertainty is embedded into problem. Notation:

\begin{align*}
    \mathbf{s} &= \begin{bmatrix}
        s_1, s_2, \dotsc, s_n
    \end{bmatrix}^\intercal && \text{Supply cost vector} \\
    \mathbf{x} &= \begin{bmatrix}
        x_1, x_2, \dotsc, x_n
    \end{bmatrix}^\intercal && \text{Procurement quantity vector} \\
    \mathbf{x^{(\max)}}  & && \text{Procurement quantity limits vector} \\
    \mathbf{x}^{(b)}  & && \text{Backlogging quantity vector} \\
    \mathbf{x}^{(h)}  & && \text{Holding quantity vector} \\
    \mathbf{d} &= \begin{bmatrix}
        d_1, d_2, \dotsc, d_n
    \end{bmatrix}^\intercal && \text{Demand random vector} \\
    b & && \text{backlogging cost} \\
    h & && \text{holding cost} \\
    n & && \text{number of time moments}
\end{align*}

\unsure{Every variable for backlogging and holding vector seems unintuitive or taken. I hope $\mathbf{x}^{(b)}$ is alright since relation to $\mathbf{x}$ is clearly highlighted}

\subsection{Variables}
\label{sub:Variables}
$\mathbf{x}$ is our decision variable, as described previously. $\mathbf{x}^{(b)}$ and $\mathbf{x}^{(h)}$ are backlogging and holding variables respectively backlogging variables respectively. For simplicity
$x^{(h)}_0$, $x^{(b)}_0$,  $x^{(h)}_n$, $x^{(b)}_n$ are equal to $0$ unless otherwise noted. This specific values are explored further in Section~\ref{sec:Variants}.

\subsection{Constraints}
\label{sub:Constraints}
\begin{align*}
    x_t &\le x^{(\max)}_t & \forall t\\
    x_t &\ge 0 & \forall t\\
    x^{(b)}_t &\ge 0 & \forall t\\
    x^{(h)}_t &\ge 0 & \forall t\\
    x_t + x^{(h)}_{t - 1} + x^{(b)}_{t} &= d_t + x^{(h)}_t + h^{(b)}_{t - 1} & \forall t \\
    x^{(h)}_0 + x^{(b)}_n + \sum_{t=1}^n{x_i} &= \sum_{t=1}^n{d_i} + x^{(b)}_0 + x^{(h)}_n &\\
\end{align*}

\section{Objective function}

\begin{definition}{$c(t)$}
defines total speeding we pay at time $t$.
    \begin{equation}
        \label{eq:cost-t}
        c(t) = s_t x_t + b x^{(b)}_t + h x^{(h)}_t
    \end{equation}
\end{definition}

\begin{definition}{$f$}
    is objective function for this problem. Our aim is to minimize it.
    \begin{equation}
        f =  \sum_t{c(t)} = \sum_t{ s_t x_t + b x^{(b)}_t + h x^{(h)}_t}
        \label{eq:cost-f}
    \end{equation}
\end{definition}
